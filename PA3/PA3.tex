%% LyX 2.3.3 created this file.  For more info, see http://www.lyx.org/.
%% Do not edit unless you really know what you are doing.
\documentclass[english]{article}
\usepackage[latin1]{inputenc}
\usepackage{geometry}
\geometry{verbose,tmargin=1in,bmargin=1in,lmargin=1in,rmargin=1in,headheight=0in,headsep=0in}
\synctex=-1
\usepackage{color}
\usepackage{babel}
\usepackage[unicode=true]
 {hyperref}

\makeatletter

%%%%%%%%%%%%%%%%%%%%%%%%%%%%%% LyX specific LaTeX commands.
%% Because html converters don't know tabularnewline
\providecommand{\tabularnewline}{\\}

%%%%%%%%%%%%%%%%%%%%%%%%%%%%%% Textclass specific LaTeX commands.
\newenvironment{lyxcode}
	{\par\begin{list}{}{
		\setlength{\rightmargin}{\leftmargin}
		\setlength{\listparindent}{0pt}% needed for AMS classes
		\raggedright
		\setlength{\itemsep}{0pt}
		\setlength{\parsep}{0pt}
		\normalfont\ttfamily}%
	 \item[]}
	{\end{list}}

\makeatother

\begin{document}
\begin{center}
\textbf{\large{}CSCE 221 Cover Page}\\
\bigskip{}
\par\end{center}

First Name~~~~~~~~~~~~~~~~~~~~~~~~~~~~~~~~~~Last
Name ~~~~~~~~~~~~~~~~~~~~~~~~UIN~~~~~~~~~~~~~~\bigskip{}

User Name ~~~~~~~~~~~~~~~~~~~~~~~~~~~~~E-mail
address~~~~~~~~~~~~~~~~~~~~~~~~~~~~~~\medskip{}

Please list all sources in the table below including web pages which
you used to solve or implement the current homework. If you fail to
cite sources you can get a lower number of points or even zero, read
more on Aggie Honor System Office website: \texttt{\href{http://aggiehonor.tamu.edu/}{http://aggiehonor.tamu.edu/}}\medskip{}
\medskip{}

\noindent \begin{flushleft}
\begin{tabular}{|c|c|c|c|c|}
\hline 
Type of sources  & ~~~~~~~~~~~~~~~~~~~~~~~ & ~~~~~~~~~~~~~~~~~~~~~~~~ & ~~~~~~~~~~~~~~~~~~~~~~~ & ~~~~~~~~~~~~~~~~~~~~~~~\tabularnewline
 &  &  &  & \tabularnewline
\hline 
People &  &  &  & \tabularnewline
 &  &  &  & \tabularnewline
\hline 
Web pages (provide URL)  &  &  &  & \tabularnewline
 &  &  &  & \tabularnewline
\hline 
Printed material &  &  &  & \tabularnewline
 &  &  &  & \tabularnewline
\hline 
Other Sources  &  &  &  & \tabularnewline
 &  &  &  & \tabularnewline
\hline 
\end{tabular}
\par\end{flushleft}

\medskip{}
\medskip{}

\noindent I certify that I have listed all the sources that I used
to develop the solutions/codes to the submitted work.

\noindent \emph{On my honor as an Aggie, I have neither given nor
received any unauthorized help on this academic work}.

\bigskip{}
\bigskip{}

\begin{tabular}{cccccc}
Your Name  & ~~~~~~~~~~~~~~~~~~~~~~~~~~~ &  & ~~~~~~~~~~~~~~~~~~~~~ & Date  & ~~~~~~~~~~~~~~~~~~~~\tabularnewline
\end{tabular}\pagebreak{}
\begin{center}
\textbf{\Large{}Assignment~3 (100 pts)}{\Large{} }{\Large\par}
\par\end{center}

\begin{center}
\textbf{Program: Due March 29 at 11:59 pm }
\par\end{center}

\noindent \textbf{Objectives: Programming}

Write a C++ class to create a binary search tree. Your program should
empirically calculate the average search cost for each node in a tree
and output a tree, level by level, in a text format.
\begin{enumerate}
\item \textbf{Programming (80 points)}.

Write code for a Binary Search Tree implementation and submit it to
Mimir. The BSTree class already has several functions that you can
use in your own functions and \textbf{you may create additional helper
functions if you find them convenient}. You must implement the following
operations:
\begin{itemize}
\item Destructor: Deallocates the memory of the tree using the delete keyword
on each node of the tree. This can be done in multiple ways (e.g.
use a recursive function, BFS or DFS).
\item Copy constructor \& assignment operator: given a BSTree object, create
a copy of the tree without modifying the given tree. For the copy
assignment operator, you must also check if you are trying to copy
the tree into itself, in which case you do nothing, or if there is
a tree in the destination and if so, delete it.
\item Move constructor \& assignment operator: given a BSTree object, move
the contents of the tree in $O(1)$ time, and empty the original tree.
For the move assignment operator, you must also check if you are trying
to move the tree into itself, in which case you do nothing, or if
there is a tree in the destination and if so, delete it.
\item Insert: This function adds a new node to the tree with the value given,
increments the size of the tree, and returns a pointer to the new
node. The new node must be given a search cost, which is the number
of comparisons required for searching a node (i.e. the number of comparisons
= the search cost for the node = $1+$ the depth of the node). Do
not use the update\_search\_costs for this. You may assume that all
the values inserted are unique.
\item Search: This function returns a pointer to the node of the tree with
the value given. If no node contains such value, return a null pointer.
You may assume that all the values on the tree are unique.
\item Update search costs: This function updates the search costs for all
the nodes on the tree. The search cost is the number of comparisons
required for searching a node (i.e. the number of comparisons = the
search cost for the node = $1+$ the depth of the node). Do not call
this function when inserting an element as this will hurt the time
complexity of insertion.
\item Inorder traversal: Traverse and print the nodes of the tree on an
inorder fashion, i.e. first print the left subtree of a node, then
the value of the node and finally the right subtree. If this is done
correctly, it should display the values in ascending order. Use the
output operator for nodes and add a single space between nodes and
it should have no newlines. See the example below for reference.
\item Level-by-level traversal: Traverse and print the nodes of a tree in
a level by level fashion where each level of the tree is printed on
a new line. Use the output operator for nodes and add a single space
between nodes of the same level. See the example below for reference.
\end{itemize}
In addition to these functions, you must also ensure that there are
no memory leaks and you must provide a makefile. This makefile should
create an executable called ``run-trees'' and use the BSTree\_main.cpp
and BSTree.pp source files. The files to be sumbitted to Mimir are:
\begin{itemize}
\item BSTree.cpp
\item BSTree.h
\item Makefile (The makefile must use BSTree\_main.cpp, but you should NOT
include it in your Mimir submission. The final executable must be
called ``run-trees'').
\end{itemize}
\end{enumerate}
\textbf{Report (20 points)}

Write a brief report and submit it to Canvas. The report should include
the following sections:
\begin{enumerate}
\item A description of the assignment objective, how to compile and run
your program, and an explanation of your program structure (i.e. a
high level description of the functions or classes in your code).
\item A brief description of the data structure you create (i.e. a theoretical
definition of the data structure and the actual data arrangement in
the classes).
\item A description of the implementation of\textcolor{black}{{} (a) individual
search cost and (b) average search cost. Analyze the time complexity
of the functions that (a) calculate the individual search cost and
(b) }sum up the search costs over all the nodes. The recurrences/runtime
functions should be from your functions (insert() for an individual
search cost).
\item \textbf{Give an individual search cost in terms of }\textbf{\emph{$n$}}\textbf{
using big-O notation}. Analyze and give the average search costs of
a perfect binary tree and a linear binary tree using big-O notation,
assuming that the following formulas are true (\emph{$n$} denotes
the total number of input data). To be clear, part 3 asks you to analyze
the running time of the functions implemented to compute the individual
and average search cost, while here you must analyze the asymptotic
behavior of the values of the search cost itself (Hint: depth of the
tree affects the individual search cost.)

Formula for perfect binary tree: $\sum_{d=0}^{\log_{2}(n+1)-1}2^{d}(d+1)\simeq(n+1)\cdot\log_{2}(n+1)-n$

Formula for linear binary tree: $\sum_{d=1}^{n}d\simeq n(n+1)/2$

where d represents the depth of the tree.
\item Use BSTree\_main.cpp to run your code to analyze the data files provided.
In case you are unable to compile and run code outside of Mimir, contact
your TA for assistance.
\begin{enumerate}
\item The files \emph{1p}, \emph{2p}, ..., \emph{12p} contain $2^{1}-1$,
$2^{2}-1$,..., and $2^{12}-1$ integers respectively. The integers
make 12 \textbf{perfect binary trees} where all leaves are at the
same depth. Calculate and record the average search cost for each
perfect binary tree. 
\item The files \emph{1r}, \emph{2}r, ..., \emph{12r} contain same number
of integers as above. The integers are randomly ordered and make 12
\textbf{random binary trees}. Calculate and record the average search
cost for each tree. 
\item The files \emph{1l}, \emph{2l}, ..., \emph{12l} contain same number
of integers as above. The integers are in increasing order and make
12 \textbf{linear binary trees}. Calculate and record the average
search cost for each tree. 
\item Include a table and a plot of the average search costs you obtain.
The x axis should be the size of the tree and the y axis should be
the average search cost. In your discussions of experimental results,
compare the curves of search costs with your theoretical analysis
that is derived above.
\end{enumerate}
\end{enumerate}
\textbf{Examples:}

Input data:
\begin{lyxcode}
5
\begin{lyxcode}
3

9

7

\textcolor{black}{10}

\textcolor{black}{11}
\end{lyxcode}
\textrm{Create a binary search tree and provide information about
each node when you display the tree. }

\begin{tabular}{cc}
\texttt{Key} & \texttt{Search Time}\tabularnewline
\texttt{5} & \texttt{1}\tabularnewline
\texttt{3} & \texttt{2}\tabularnewline
\texttt{9} & \texttt{2}\tabularnewline
\texttt{7} & \texttt{3}\tabularnewline
\texttt{10} & \texttt{3}\tabularnewline
\texttt{11} & \texttt{4}\tabularnewline
\end{tabular}

\textcolor{black}{Total number of nodes is 6}

\textrm{The inorder traversal for this particular tree is:}

3{[}2{]} 5{[}1{]} 7{[}3{]} 9{[}2{]} \textcolor{black}{10{[}3{]} 11{[}4{]}}

Here~The~format~of~each~node~is~value{[}search~time{]}

Sum of the search cost over all the nodes in the tree is: $2+1+3+2{\color{red}{\color{black}+3+4=15}}$. Average
search cost: ${\color{red}{\color{black}15/6=2.5}}$.

Average search cost is \textcolor{black}{2.5}

\textcolor{black}{Output the tree level-by-level to a file (missing
elements are denoted by X):}

\textcolor{black}{5{[}1{]}}

\textcolor{black}{3{[}2{]} 9{[}2{]}}

\textcolor{black}{X X 7{[}3{]} 10{[}3{]}}

\textcolor{black}{X X X X X X X 11{[}4{]} }
\end{lyxcode}
\textbf{\textcolor{black}{Hints}}
\begin{enumerate}
\item \textcolor{black}{Besides using links/pointers to represent a binary
search tree, you may store the binary tree in a vector. This implementation
might be useful, especially for the printing of a tree level by level.}
\item You can add your own recursive functions on the header file as long
as you define them in the cpp file and you don't remove or change
any of the functions that are defined already on the header.
\item \textcolor{black}{You may use the std::queue and std::stack classes
to perform BFS or DFS repectively}
\item \textcolor{black}{The pseudocode here is one way of doing the level-by-level
function. You can create your own version if you find it easier. }
\begin{lyxcode}
level\_by\_level(BinarySearchTree~T)

~~~Queue~q~//~This~queue~contains~elements~from~a~level~and~its~children
\begin{lyxcode}
q.enqueue(T.root)

elementsInLevel~=~1~//~Elements~in~the~current~level

nonNullChild~=~false

while~(elementsInLevel~\textgreater ~0)~do:

~~~~TreeNode{*}~node~=~q.dequeue()

~~~~elementsInLevel-{}-

~~~~if~node~is~not~null:

~~~~~~~~print~node

~~~~~~~~enqueue~the~children~of~node~into~q

~~~~~~~~if~at~least~one~child~is~not~null:

~~~~~~~~~~nonNullChild~=~true

~~~~else:

~~~~~~~~print~'X'

~~~~~~~~enqueue~null~//~these~nulls~represent~the~descendants~of~the~empty~node

~~~~~~~~enqueue~null

~~~~if~elementsInLevel~==~0~//~We~have~reached~the~end~of~the~current~level

~~~~~~~~print~newline

~~~~~~~~if~nonNullChild~==~true:~//~The~next~level~is~non-empty

~~~~~~~~~~~~nonNullChild~=~false

~~~~~~~~~~~~elementsInLevel~=~q.size
\end{lyxcode}
\end{lyxcode}
\end{enumerate}
\noindent \begin{flushleft}
\texttt{\smallskip{}
}
\par\end{flushleft}
\end{document}
